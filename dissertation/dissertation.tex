% This example An LaTeX document showing how to use the l3proj class to
% write your report. Use pdflatex and bibtex to process the file, creating 
% a PDF file as output (there is no need to use dvips when using pdflatex).

% Modified 

\documentclass{l3proj}

\begin{document}

\title{QPQ Game}

\author{Almohannad Albastaki \\
        Ronaldas Gadzimugometovas \\
        Tom Gough \\
     	Conor McHarg \\
        Borislav Hristov}

\date{31 March 2017}

\maketitle

\begin{abstract}

The abstract goes here

\end{abstract}

%% Comment out this line if you do not wish to give consent for your
%% work to be distributed in electronic format.
\educationalconsent

\newpage

%==============================================================================
\section{Introduction}
\label{sec:introduction}

This document presents the dissertation of Team B’s, a team consisting of 3 third year Electronic and Software Engineering and 2 Mobile Software Engineering students, financial analysis tool. The tool was developed for the Adam Smith Business School (ASBS) to act as a financial aid for their Quid Pro Quo (QPQ) simulation game. The primary aim of the tool was to provide an efficient and simple manner of processing, storing, and exporting relevant data; output from the simulation game.

    The document will delve into software development processes practised by Team B throughout the project and how these practices affected the team’s decision making, planning, and development techniques. Through this case study, Team B aims to achieve a better understanding of why these processes were taught in the Professional Software Development course at the University Of Glasgow, and how to better implement them in future team, or individual, projects.

    The QPQ game itself is a hands-on paper-based simulation game which mirrors a typical production line found in most modern organisations. Participants of focus will usually be second year business students. Students are required to complete “orders”, with appropriate timeliness, to see a profit. When the game comes to a halt a financial analysis, from the resultant data, is conducted. The main goal of the simulation game is to teach these Business students the importance of employing various business process theories, and how these processes affect any given production based organisation.  

The dissertation is structured as follows:

Section \ref{sec:casestudy} - Case Study Background - outlines the case study in more depth. Details of the customer, preliminary objectives, stakeholders, team aims, initial team planning, and the final product are further discussed in this section. This section also mentions the workings of the game, as interpreted by the team, and the key solution to the issue faced while playing.

Section 2 -  Team Dynamics - details the team composition and dynamic. The team had opted for a laissez faire styled leadership structure, where each member has equal decision power. The benefits/consequences of this and how it affected the development process are reflected upon throughout this section.

Section 3 - Group Meetings, Client Meetings, Communication, and Planning - discusses the structure and organisation of group and client meetings as well as the communication techniques involved out of meeting. Various planning techniques, such as wireframes and class diagrams, were employed throughout the project and their effects are discussed further here.

Section 4 - Software Processes and Prototyping - will directly pick out the development techniques utilised in this project. Such techniques include, but are not limited to: Peer programming and Sprints. In particular, an evaluation of the improvements seen from employing these practises and the benefits they provided towards prototyping will be discussed.

Section 5 - Testing, Continuous Integration and Tracking System - explains the integration mechanisms and techniques. Specifically, a free web-based Git service (GitLab) was employed to provide the team with all its tracking needs. An account of the benefits and drawbacks of this service/tracking mechanism is presented as well. Furthermore a detailed account of the testing process and the limitations it posed are identified here.

Section \ref{sec:development} - Code Development - elaborates on the front-end and back-end development. This will involve a discussion with regards to the technology choices made and the affects these decisions had on the project. The greatest challenge here was the learning curve involved in the technology used and trying to implement the required functionality.


%==============================================================================
\section{Case Study Background}
\label{sec:casestudy}

\subsection{Customer}

The Adam Smith Business School (ASBS), a professional research institution committed to the field of business and economics, situated at the University Of Glasgow, had developed a simulation game, Quid Pro Quo (QPQ), to be utilised during business tutorials to give students a hands on experience of the workings of the production line, to ultimately supply a customer with the required demand.

The QPQ simulation game is the brainchild of Dr Rob Dekkers, a Reader in Industrial Management at the ASBS, and his team of PhD research students. Dr Dekkers has written publications on the concepts of Bottlenecks, Engineering Processes, and Production Management. The solution to the simulation game consists of following a number of business process theories from these focuses. The details of this will be further discussed in the dissertation. 

Throughout the project our main point of contact was Dr Rob Dekkers himself, however often accompanied by his research students; Maria Koukou, Mohammed Aldossary, and Qijun Zhou; All of whom often provided useful information of business theories during the meetings.


\subsection{Preliminary Objective and Stakeholders}
The research students of Dr Dekkers, not able to attend himself, had conducted the first introduction of the game and issues they faced. The QPQ simulation game is a paper based game which reflects a real life production line. The idea behind the game is that a certain amount of orders for cars (lego cars in the simulation game) are made by a game leader, usually a PhD student. These orders, which currently comprise of multiple sheets of paper, are passed between students (employees), who act in accordance to their assigned departments (e.g. Production Planning, Good Receipt, Sales) all controlled by a hands-off manager, who is a student him/herself. 

The game consists of Three 20 minute rounds in which employees must complete all orders with timelines to see a profit. The manager, and fellow employees, are given the opportunity to discuss solutions, various practices, and even rearrange themselves to achieve the desired profit at the end of each round. External (AYN) employees can be hired to assist employees in their respective departments before and during rounds. The AYN in most tutorials conducted will be the PhD students, who have the expertise and knowledge behind the game.

At the end of each round a financial report is produced. This was stressed to be a very cumbersome effort as it involved looking through many forms, previously filled in by employees, and producing calculations from the variables gathered. The financial department is usually a collective effort of a student, the manager and a PhD student, to assist. Per round the financial department would be required to look through the multiple orders produced and calculate a revenue and various, specific, expenditure figures to, hence, calculate the profit/loss seen in that round. On average the financial department will face about 20+ filled in forms. Due to, primarily, this activity the game sees many delays and students are often left waiting 15-20 minutes between rounds.

The whole game would last around 2 hours, which to Dr Dekkers was not a satisfactory run time for only 3 rounds. As a result of the wasted time from the financial stage, tutors are forced to reveal the solution in a quick, sometimes unclear, manner. In his view, the students, who were seen as the main stakeholders for the project, must be given more time within the rounds to be able to solve its workings themselves. This meant striving towards greater amount of rounds per tutorial, and secondarily, providing a means for keeping figures and variables coherently documented for later study by all users of the game.

The solution to the game involves employing Enterprise Resource Planning (ERP) and spotting the location of a bottleneck within the production line to properly see a profit. A bottleneck being a point where activities slowdown due to a department working at full capacity. In further detail, the source of the problem is the lack of building components at the production phase, this is the location of the bottleneck. By recognising this bottleneck and employing ERP, employees, will be required to supply the production department in a just in case manner, where common building components are supplied before an order for a car consisting of this component is made, to see an overall profit for the round.

\subsection{Aims}
Dr Dekkers and his team had taken the stance of allowing the developers to use their creativity to developing the products functionality as per the initial introduction and a game demo provided. With regards to the issues mentioned on the financial analysis, it was seen that reducing this time was key. The main challenge here was translating human processes into a computation and furthermore visualising this in a readable and coherent manner. It was however stressed that an export functionality was to be produced out of these computations, which essentially meant data storage and file writing capabilities.
    
It was as well mentioned that there was no guarantee of a network connection during some tutorials and due to the various computer and smartphone platforms used by students and tutors; a consensus was made that a cross computer platform application was to be developed. This meant developing on a single language that was able to execute itself on any platform.
Ergo, to define the primary aim, it was decided upon that a computer application to mimic the financial departments forms/activities would best deal with the game's timing and documentation issues without interfering too much with the hands-on experience it currently provides.


\subsection{Initial Planning}
Initial requirements gathering involved a full demo of the game. It was here that team members got themselves involved directly in the game, with assistance of ASBS students and another TDP 3 team. Each team member was to keep note of the workings and difficulties they faced in their individual departments. Further qualitative data on the workings of the game was collected by a single member who watched over the play through. At round end, the financial analysis process was conducted and studied. A record of the time it took and arithmetic conducted was taken.

The QPQ simulation game framework (i.e. every form type as specified to department) was provided to the team. This was the main basis of the back-end computational software and gave an idea of the front-end user interface. Over the previously mentioned readability/coherency requirement; the front-end user interface was to be developed so that students may pick up the application without any need for many instructions except the ones already provided by Dr Dekkers’ team. Wireframes were consequently developed from this documentation and hence the application would consist of various screens which provided the service as specified by the respective forms they mirror. The back-end software on the other hand had the requirement for quickly computing the input data and storing this data for export. The documentations conveniently provided the organisation scheme for the data storage and the connections between each storage unit; as well as the arithmetic required behind each generated value. A report was to be consequently generated from the input data and resulting output data.
    
A MoSCoW prioritisation scheme was employed to pick out the forms required for the application. This meant splitting the features into must have, should have, could have, and won’t have but would like categories. As we had to maintain the scope of the project to fit within the time frame of the TDP 3 project a few forms crucial to the financial team were selected to be developed, discussed in section .6 (Final Product Features).


\subsection{Development and Further Requirement Gathering}
Throughout development regular meetings were held to discuss development progress and approach for future prototypes. As the program had a fairly simple and clear structure; functionality for each form was delegated amongst team members, accordingly. Separate components were to later be linked together. The choice of software language and supporting mechanisms/libraries was key to accommodate for this development approach.

A top down development approach was agreed upon, where the basic look of the application would be developed followed by the desired functionality. Each team member was responsible for initially organising the layout of the screen, then using the present textfields and labels to manipulate the relevant input data in accordance to the form assigned. An export functionality was  developed after these components were linked and it was confirmed that data was moving around correctly. At the same time styling components were added as per the client's request. The styling component was to abide by the University’s/School’s colour scheme and include the ASBS logo.

Another notable request was for a settings page where game rules may be amended. This was to be password protected, so only Dr Dekkers may alter this information. The settings page was to alter details such as wage schemes, tardiness penalties, material prices, and even game time. A later request was also made for a count down timer to show the progress of a round to participating students.  These particular features were seen as supplementary to the core functionality of the program  and were incorporated into the final product at  the end of development.

\subsection{Final Product Features}
Following the QPQ game framework a financial assistant tool was produced to tackle the client's timing problem during post round financial evaluation. The lightweight tool can perform all the computations/arithmetic required by the financial team. The application itself does not interfere with the workings of the game but only hastens the activity of the financial team.

The computational aspect is considered the main feature of the application. As mentioned each screen mimics a form seen on the QPQ game framework. The main screen (Form V) lists all the output data for the processed input data. Users are required to access each field by clicking on its respective button to access another screen (form). Within each screen there are fields for the forms required input data. Form B, where a certain order’s financial details are input, such as contract price, and scheduled delivery time. This form ultimately produces total revenue seen in the round. Another screens pertains to the employee details (Form S). This form only requires the employee name and position, the application will compute the total wage from these details. Form O represents a summary of the orders input in form B and computes the total price of the materials used. Form T is dedicated to holding the AYN employees details, who have their own unique wage scheme. Form V will, finally,  compute the profit/loss seen by that round in accordance to the above produced figures. A StartScreen has also been provided to act as a lobby between rounds. This screen consists of a start button, which generates a new round, a setting button, to access the setting page, and a report generation button for export, as well as a timer button , to open a timer in a separate window, as per requested by the client.

The simplicity of the tool and its close similarity to the already existing documentation, means it requires little extra user instructions. The extent of which consists of the purpose of each form and their meaning as described by the game documentation provided by Dr Dekkers. The application itself is small in size, portable via a physical storage unit, and has no network connectivity requirement. This means the tool is quick, simple to setup and utilise during intensive tutorials
%==============================================================================
\section{Group Meetings}
\label{sec:groupmeetings}
\subsection{Requirements Discussion and Delegation}
Group meetings were a key element in the success of our project. The primary role of these meetings was discussions on the requirements of the project and to delegate tasks to team members. This aspect of the development process took up an extensive amount of our time. Initially our meetings lacked structure and this lead to the discussions occasionally focussing on low priority problems or going off topic. However we quickly identified this issue through a retrospective we carried out. To tackle this problem we started to identify key talking points for the meeting at the start of each meeting, allowing us to effectively focus our attention on what we believed to be the priorities of the project. As a result our meetings became much more productive and we were able to focus more time on actual development of the application. This method of keeping team meetings focussed on the important issues was effective, however in retrospect making an agenda prior to each meeting would have been the best solution for keeping meetings structured.

\subsection{Progress Discussion}
The meetings were also used to track the progress of the project and to plan the next sprint. As each team member was developing different parts of the application separately, reporting the progress of each member was important for keeping the project on track and effectively planning what had to be done. Planning what was to be done in the next sprint and delegating these tasks was also carried out during these group meetings. Making the decisions about task delegation as a group allowed us to effectively take advantage of each team member’s strengths and ensure everyone was comfortable with their workload. In the early stages of the project the tasks set out for each sprint were fairly large and vague. As a result few of the objectives for the early sprints were ever met. We identified this issue early on and quickly started making sprints made up of many small and specific tasks. This way each member could be allocated several small tasks in each sprint, making it easier for the team to prioritise important tasks and ensure they were completed by the deadline set. This demonstrated to us as a team that sprints made up of small, easily achievable tasks are much more effective for making good progress than sprints with fewer, larger tasks.

%==============================================================================
\section{Client Meetings}
\label{sec:clientmeeting}
\subsection{Requirements Elicitation}
The team met with our clients on various occasions for various different reasons. The main focus of our meetings, especially in the early stages of development but continuing throughout, was requirements elicitation. As mentioned previously the team took part in a live demo fo the game to gain a better understanding of the requirements. In every meeting following the demonstration of the game, new requirements were gathered for the development of the app. This ongoing requirements gathering continued up until the final iteration of the projects. An issue the team faced was clarifying requirements with the customer as they were not always sure what they wanted from the application. This made the continual requirements gathering essential as the customer’s view of what the app should be, changed throughout development.

\subsection{Progress Reporting}
Reporting the development teams progress to the customer was primarily done through face to face meetings. As previously mentioned the team faced difficulties in ensuring the time spent meeting the customer was used effectively. In part this was due to the fact that in the early stages of the project little progress was made that was easily demonstrable to the customer. For example much of the development time was used to further understand the game and the requirements of the customer in addition to generating prototypes and concept designs. As discussed above the team adopted a top down design approach to overcome this issue. This allowed us to us to not only give the customer an idea of what the application would be like to use, but also to demonstrate further work done on the back end of the application later in the project. This highlighted to the team the importance and effectiveness of visual aids when dealing with and reporting a development project’s progress.

\subsection{Project Scope}
One of the key decisions on the design of the application and an early focus of our client meetings was the scope of the project in terms of both how it would be deployed. Initially the team had several different ideas about how the project could be deployed, after meeting with the customer this was narrowed down to two options. A mobile based application running on a web based server or a localised application designed to run on a single PC. The team believed that as a software solution the web based app would be a better choice as it would vastly improve the ease with which the game could be set up and run allowing the students playing to focus on the key points of the game. However through many discussions with the client it became clear that they did not want to rely too heavily on technology and did not want an application that would rely on an internet connection to function. So the decision was taken to develop a localised application that could be run on one laptop and would focus on one of the most time consuming aspects of the game, the financial analysis. The customer much preferred this plan as it provided them with an improvement to the game without making it reliant on technology. This was an excellent demonstration of an instance where the development teams idea of the best solution to the problem is not necessarily what the customer wants from the product. In this case the team must align their objectives with the customers. The decision the team made on the project scope was taken after several discussions with the customer in which multiple different proposals were put forward until the team settled on an idea the customer was happy with and that they felt would fulfil their requirements.
%==============================================================================
\section{Communication}
\label{sec:communication}

\subsection{Group Interaction}
Our main means of communication were face to face meetings and through social media. We had a facebook group which we regularly used to organise meetings and to report progress made in between meetings. However it was in face to face group meetings that most of the discussion and communication between team members was done. It was in these meetings that the team made the most progress on development of the application and it was identified in a retrospective we carried out that having more group meetings would significantly benefit the development effort. This allowed developers working on different elements of the application to more closely coordinate their work making the integration of different parts of the app much easier. These group coding meetings were highly productive and in hindsight the team should have exploited this development technique much sooner in the development process.

\subsection{Client Communication}
All communication with the client outside of meetings was done through email. The team set up a joint email account which we used to contact the customer between meetings. However one of the main issues we faced was the promptness of the clients reply. Our customer was always very busy and as a result getting in contact with him proved difficult. To mitigate this problem the team learned to email as far in advance as possible so as to increase the chance of getting a timely reply. In addition to this an effort was made to look ahead in project development so that issues that may arise in future between meetings could be discussed in earlier meetings while we had the clients time. This issue highlighted the importance of using face to face meetings with the customer as effectively as possible so as to reduce the need for communication between meetings. Through a retrospective the team identified more face to face meetings with the customer as something we should pursue and so an effort was made to organise more regular meetings.
%==============================================================================
\section{Planning}
\label{sec:planning}

\subsection{Class Diagrams}
The team made use of several planning strategies during the course of the development process. One of the early techniques we used was creating class diagrams. As the application had many classes that needed to interact with each other and share data there was a need for a structured plan of what the architecture of the program would be. Class diagrams met this need.

Shown above is one of the rough class diagrams the team drew up in the early stages of development. This diagram and others liked it enabled the team to effectively plan the design of the application.

\subsection{Wireframes}
After the second meeting with the client the team chose to create some simple low fidelity wireframes (shown below) so that we could demonstrate to the customer our idea of what the application would look like and get their feedback on the design. This technique enabled each team member to showcase their ideas and allowed us to take the best ideas from each design and merge them for the wireframe shown to the customer. It was then agreed in a meeting with the customer that a medium to high fidelity wireframe should be drawn up so that both the customer and the team could get a better idea for the desired look and feel of the application.

\subsection{Development Estimation}
Estimating the time taken to complete different stages of the development was one of the major challenges the team faced. Quite early on in the project the team decided to use sprints as a development technique. These sprints were usually about a week in length during which each member would be assigned multiple small achievable tasks to complete by the end of the sprint. Our initial sprints however never met all of  their objectives. This was mainly due a lack of experience of  both developing this kind of application and of small team development resulting in the team overestimating what could be achieved in a week. At the end of each sprint the team reviewed the sprint and it soon became apparent that we needed to better assign tasks. As a result in latter stages of development the team often met most if not all of  the objectives set out at the start of each sprint. The use of sprints significantly benefitted the projects productivity, it allowed the team to break down the large task of developing an application from scratch into small and manageable tasks enabling steady incremental progress.

\subsection{Peer Programming}
One practice which all team members engaged with was peer programming. Peer programming involved usually 2 team members who would work together on one particular problem, one coding and the other watching and suggesting improvements. This was normally done during team programming sessions, in particular, during sprints to help tackle difficult tasks, catch typing errors and suggesting debugging choices. During these sessions we were able to solve problems much quicker and make very good progress adding additional features nearer the end of development. One area where peer programming worked very well was understanding and developing on another team member’s code. With different coding styles within the team, peer programming helped by having another option on the code that had been written and the code that was being written together. 
As a team we didn’t start peer programming until the first sprint as we had mainly been coding individually. Once we realised how much progress was made working together, peer programming became a regular occurrence during sprints and often even after team meetings when everyone was together to help each other work through their individual problems.


\subsection{Team Dynamics}
The project brought together two Mobile Software Engineering and three Electronics and Software Engineering students. 
Each participant proved to have specific skills, relevant to the project and preferences, in terms of tasks and the project as a whole. 
The differences in terms of the study programs did not prove to be a factor.
 
When it comes to organization, team members had equal decision making power. 
The team chose to avoid hierarchy within as no member had considerably more expertise than the others. 
On the other hand, individual members would step up when feeling that their contribution was valuable and step down when the others seemed more knowledgeable. 

Even though the team was well balanced and had little if any problems working together, there were a few things that hindered the effectiveness of the team. 
First, each team member had their own personal goals, preferences, 
and biases which were given priority over the project at some stage or another. 
For example, in the beginning of the course the team was more eager to push their own ideas and design goals to the customer than designing in response to customer needs. 
However, due to continuous progress self evaluation the team noticed the lack of effectiveness and measures were taken to avoid this, 
namely documenting precise requirements and ticking them off once addressed. 

A similar issue was that each member had different coding experience and coding style. 
Things like preferred coding strategies, choice of data types, code readability, etc. caused the team to be inefficient. 
Team coding process involved the members working on different parts of the program separately first and then the group would integrate the work. 
Unfortunately, given the differences mentioned above, large portions of good working code needed to be modified or completely replaced in order to make 
all integral parts compatible with each other. 

Naturally, the team took time to consider possible ways of minimizing the effect of team differences in further development. 
One of the strategies included careful allocation of tasks which would minimize integration effort. Furthermore, the team agreed to put additional effort 
foreseeing potential integration issues (e.g. variable naming) and preparing themselves (e.g. agreeing on the shared variable names in advance).

\subsection{Testing}
The team ensured sufficient testing by devising the following strategy. First, since code errors of GUI development were clearly visible upon running the 
software, the team required no specific test cases. On the other hand, each team member was required to run the program and navigate the GUI until certain 
that implementation was correct before pushing their code to git. Additionally, each team member was required to run the program and make sure that it ran 
as expected immediately after pulling the updates from git. Such strategy allowed the developers see which code is correct and which parts of GUI still 
required attention.

In support to GUI testing method, the team held several code review session, where team members had a chance not only to familiarize with each other’s code 
but more importantly identify bugs and general code weaknesses. Moreover, the team was able to give suggestions for improvements. The code reviews did not 
prove to be very effective finding issues in the code as again, given the nature of the project, the GUI would either be working correctly or not working 
at all. On the other hand, code review proved to be extremely helpful assisting the members who were stuck or generally sought advice.

The second step in terms of testing was developing JUnit test cases to identify existing functionality bugs. Since the product was characterized by only 
basic data analysis (adding and subtracting numbers), only several JUnit test cases were created. Nevertheless, regardless of the limited amount of test 
cases, the test suite helped to identify and eliminate a number of calculation and data retrieval bugs. Therefore, JUnit testing proved itself to be an 
inseparable part of product development.

Lastly, it was negotiated with the customer to have a full run-through of the game to test the completed product in action. The real-time product testing 
proved to be exceptionally valuable. 
Not only the team was able to see whether the software managed to perform in game tasks but also observe how the application coped when the game turned 
unexpectedly (i.e. was interrupted, reset, fast-forwarded, etc.). This simulation allowed the team to makes sure that both the program operated correctly 
and was flexible enough to handle unpredictable playtime issues.

\subsection{Continuous Integration}
The team used built in Gitlab integration tools. Since the members of the team were inexperienced with continuous integration tools prior to this project, 
it required some learning before the tools could be fully exploited. Inexperience and unfamiliarity were the sources of reluctance to continuously push the 
code to Gitlab. What is more, initially, the members preferred to use a messaging application to send their code directly to each other. However, it did not 
take long for the team to realize that professional software development required a more sophisticated method of code sharing.

Once the members familiarized themselves with basic git commands, they utilized branches. The code development was distributed to a small number of branches 
which were then merged to the master branch, once individual development stages were completed. The master branch served as a backup in this project whereas 
most of the development happened in a ‘development’ branch. Others were used to experiment with additional, low priority functionality.

In order to make sure that unintentional bugs would not be pushed to a branch the team utilized automatic build testing, i.e. upon pushing, the code was 
checked to see whether it compiles and only accepted if it did. The team did not use a more elaborate automated testing because as discussed in the Testing 
section, the nature of the project required different testing methods.

\subsection{Tracking}
Gitlab built in tools were used to track team progress and plan future tasks. Gitlab capability of opening issues allowed the team to create a record of the 
tasks for the project. Each issue could have custom labels attached to indicate a theme and priority. In addition, the tasks could have weights assigned to 
indicate corresponding estimated time to complete the task. Once the task was completed, the team was able to look back, evaluate the work done, and close the 
issue.

Tracking proved not to be a difficulty in the project. The team had a dedicated member to make sure that the issues were opened and maintained as required. 
However, it did take time to get accustomed to using the full range of the issue attributes (e.g. labeling). At first the team incorrectly used weights to indicate 
how important each issue was and had no labels on each issue. This was due to the lack of experience within the team using Gitlab and issue tracking. At first 
the issues the members opened were vague and had little detail in the comments which in some cases caused delays as the assignees might not be fully informed of the scope of the task.
 
Supervisor feedback caused the members to make changes to how the team used tracking and set a standard within the team of how each issue must be opened and laid out. 
Development quickly picked up after these changes as team members were able to close issues in their own time without waiting to discuss any further details. 
Labels were set up to indicate the priority and subsection of each issue. For example, ‘high priority’ and ‘code development’. The team then used weights correctly 
to indicate how we long expected each issue to take. This resulted in team members being allocated equivalent workload. Another change the team made to 
issue tracking was referencing the commit to their respective issues.

%==============================================================================
\section{Code Development}
\label{sec:development}
\subsection{Front-End Development}

To facilitate the transition from paper forms to our digital application, we decided to make the different views as close as possible to the already established forms. This way users could learn how to use the program faster and transfer the data from the paper forms to the digital ones easier. For images of the game play check appendix section \ref{sec:appendix}

\subsubsection{Settings}
After consulting with the client, we created the settings page so all the game data (kit prices, game rules, products, product features, product prices and etc.) to be editable. Each of the four forms (C, D, P and R) that the game takes data and the game rules have their own section from which they can be edited. The changes are not saved until the user is contempt with them and selects to save them. If the back button is pressed before saving made changes, the user is asked in an alert view if he/she wishes to save those changes. To ensure the security of the data, a password protection for this section was created, the user has to enter the password every time they wish to enter it.


\subsubsection{Game}
After starting a game, the user has the ability to go back to the main menu, to save and load a game, to add rounds to the current game, to generate a report for the students and to open a particular round. 
When the game first begins there is only one round added. If the user wishes he has the opportunity to add more (the maximum set of rounds for all games is set from the settings). This was done, because the customers were worried that not all games will have enough time to finish the predefined number of rounds. So instead of having a couple of pre-generated rounds, each one is constructed after selecting “Add round”. This has the bonus effect of keeping less things into memory.
As mentioned above each game has the possibility of being saved and then loaded at a different point in time. This allows the customers to stop playing a game, and later if they wish to continue playing it.
The report is one of the features the customers desired the most. This is, because each student who plays the game has to receive a copy of the final game data, and make conclusions from it. Each report is generated with the data from the different rounds in the current game. It is split into sections, each corresponding to a different form from the multiple rounds.


\subsubsection{Rounds}
Each round resembles the “form V” provided by the customer. No data is being editable from here, as this page functions as navigation to the main sub-forms filled by the user and as a representation of all the data inputted so far. This was done, to reduce the amount of data entered by the user and to remove the part where the user had to make the different calculations from the sub-forms. This was created in an effort to save time for the people using the program.

\subsubsection{Form}
As mentioned before, the forms try to resemble their paper counter parts as much as possible to reduce the time a user needs to get accustomed to them. As form B requires the most input from the user, we decided to make the process as fast as possible. This was achieved by adding auto-complete to some of the fields which require not data from the played game, but data from the system itself.

For form O all the data is generated by the system and the user does not need to input any data. This was done, because all the data required for it is already set up in form B, and we didn’t want to make people have to re-input the same information over and over again.

Form S deals with the permanently employed employees. The user can add new ones or delete old ones at will. Each employee requires a position and a name. This is needed to make the calculations for how much money they are paid in the “V form”. Employee rows, whose data is left blank on return to the previous form, are not saved in the records.

Form T deals with the employees who are not part of the company and come to help out when there is need of extra hands. Since they may not play the entire round, the user has to select for which rounds the employee has worked, on what positions was he hired and did he work in multiple departments.

\subsection{Back-End Development}
\subsubsection{Java}
When we first started to create the project, one of the first things we had to choose was the language we were going to use to create it. After some consideration we decided to use Java mostly because its cross-platform compatibility (its binaries work on all platforms with the exception of Android, which either way was not in our intended targets), since the customer wanted to be able to play the game on both Macs and PCs. Additionally since all of the team had previous experience with the language it was the wisest decision to make.


\subsubsection{JavaFX, CSS, FXML}
The graphical interface of the project was created with JavaFx. We chose it, because it introduces several improvements over Swing (such as the possibility to markup UIs with FXML and to theme them with CSS). Additionally, JavaFx has a cleaner and more consistent API across components, and while Swing is a fully featured and supported library for the moment, it may not be for the future (JavaFX is replacing Swing as Oracle’s UI library for java), which would have limited the project if additional features were required in the future.
We decided to use FXML for the project, mostly because it requires little work to implement complex views with multiple data entry points. An additional advantage to this language is that since FXML is not a compiled language, we didn’t need to reload the project every time a change to the FXML was made. This allowed for faster work process and less distractions. 
Since the basic objects created with FXML do not have a nice design, we decided it would be a good idea to add a custom one to the project. To achieve this we used CSS to create our own styles for the different pages.

\subsubsection{Singleton, Data structures and System Data}
Most views in the application are created and destroyed when the user opens and closes them. Because we needed to keep the data throughout the entire time the game was played, we decided to use a singleton. This allowed us to store all the data from the many different views, and the system data in one place. This kept it concise and allowed for quick implementation of changes. Additionally, because the singleton has only one instance, we could transfer data from the different views, without having to implement multiple functions between the different views. 
The system data is separated in the four forms (C, D, P, R) and game rules. Since these are needed throughout the entire game their information is stored and accessed from the singleton. The first time a user requests data from one of this forms, the system checks if that data has already been loaded. If it hasn’t then it opens the corresponding file in the database and loads its information into memory, then it returns the required information to the user. This was done to avoid having to open and close the database documents multiple times. 
The game rules data is also mapped into a hash table, since the user may only require a specific field from all the information in the table. The hashing is done through the same means as loading the data from the database files. The first time a specific field is called, the hash table is created with the name of the row as key and the value of the row as value for the hash. After that the specific field is returned.
All the data from the many forms is also stored in the singleton. The data from each “V form” is stored into an ArrayList. There are functions that allow the user to call on that data, update or delete it. This allows for data to be edited even if the round is closed. With the export (save) and import (load) functionality, this even allows for game data to be edited after the program has been closed and reopened.

\subsubsection{Exportability (save) and Importability (load) of a Game}
One of the clients’ requirements was that a game should be able to be saved and reloaded at a later stage. To achieve this, we used the functions for setting, getting and updating game data from the singleton to access all of the game data that was in memory to this moment, and then create four “.csv” files. We decided to save the data in this format, because we desired the client to have to ability to open a saved games data, even if they did not have the game itself in front of them. 
When saving the game data, it takes all the information from the singleton, changes it into a string (each ArrayList element represents a row in the “.csv” file) and then saves it to the designated file. The loading of data is the reverse process, with files being read line by line, and each line producing a different row in the ArrayList. There are four main files created after each game export (save). An additional feature, we implemented is that the user does not have to load all four files, but as many of them as he wishes. This allows for easier edit of the data. (i.e. load all the “V and B forms”, while not getting the employee data).

\subsubsection{Controllers}
Since most of the controllers had a number of repeated functions, we decided to create a “Base Controller”, which is extendable from all the other Controllers in the application. This allowed us to avoid redundancy in our coding, and saved us time when creating the many different pages. An additional advantage to this approach was that it allowed us to have a skeleton of each page, on which we could quickly build the functions we needed for that particular page.

%==============================================================================
\section{Conclusion}
\label{sec:conclusion}
\subsection{Conclusion}
Group software development projects tend be cumbersome and time consuming. There is a heavy emphasis on the communication of ideas to keep development intact. Agile software development processes provide useful frameworks to be able to deal with the difficulties behind group projects. The main purpose of this course was to evaluate these implemented practices while developing a functional software application to satisfy the clients’ requirements.

A notable lesson to take from this study, was the requirement to operate in a  professional manner throughout the course. Most members, prior to the beginning, had not undertaken a professional project, but rather worked individually for university assignments. The inexperience with these professional practices meant requirement elicitation, engineering, and estimation was fairly disorganised. However, after research and feedback from supervisors, it became clear to Team B how to tackle and manage development, whilst implementing agile software practices. The effects of this change was evident throughout the second half of development where clear milestones were set on a regular basis. Setting smaller detailed issues made communication more effective and development more coherent as each team member gained a better understanding of what was required of each component.

From analysing the development process as a whole, we have learnt many practices and improvements after making mistakes and learning from them. One area we would ensure was planned and designed to stricter standards is code structure. Due to each team member having different coding stylings and experience the code quickly moved away from the initial object oriented design which was first suggested. As a team we did not make strict coding rules when designing the application. In hindsight this was naive and caused some code to styled differently. A stricter design pattern, more time spent programming together, and discussion of each element would have prevented this from happening as the code would have been more organised and coherent.
%==============================================================================
\pagebreak
\section{Appendix}
\label{sec:appendix}

\subsection{User Documentation}
\subsubsection{Starting the game}
On entering the game, the user is given the option to go to the game lobby by clicking the “Start Game” button or the settings lobby by clicking the “Settings” button. On clicking the “Settings” button the user is prompted for a password (held by the game administrators).

\subsubsection{Settings}
The user, here, is given the option to edit game rules and in game data. 
By clicking “Game Rules” the user is redirected to its respective menu (shown below). After alterations are made the user must save the data by clicking the “Save” Button.
Form C, D, P, and R allow the user to edit data that will be used by the application to process input data against. On editing each Form the user must save before exiting to confirm data is changed. 
Shown below is Form C.
\subsubsection{Game Lobby}
In the game lobby, the user is presented with a number of option, shown below.
\subsubsection{Adding a round}
The user must click “Add Round” to generate a new round that will appear below the solid line as shown to the right.
\subsubsection{Saving a game}
On clicking the “Export” button 4 .csv files pertaining to data in Form B, S, T and V are generated which may later be imported through the “Import” button. No headers are present on these files.
\subsubsection{Creating a report}
The “Report” button will generate a single .csv file filled with the correct headers for all the rounds.
\subsubsection{Timer}
When the “Timer” button is clicked a new window is opened (Show to the right), showing a count up timer that will run for the amount of time given in game rules.
\subsubsection{Adding round data}
To add round data the user must click on any of the round buttons which appear below the solid line to access the round lobby (Form V, shown below).
In the round lobby the user has the option to access each form, by clicking its button, and input the data related to that form. On returning to the round lobby from each form, data is auto generated and presented.
\subsubsection{Form B}
Enter data correctly for each field. An example is given below. Scheduled Delivery Time, Chassis Type, Penalty and Revenue are automatically generated.
\subsubsection{Form O}
The user here is presented with a summary of the orders made and list the expenditure in terms of materials. Example given below.
\subsubsection{Form S}
Enter data for every employee correctly. New employees may be added by clicking “Add New” Button and existing employees may be deleted by selecting the the employee then clicking the “Delete” button. Data will save on clicking “Back”. An example is given below.
\subsubsection{Form T}
Enter data correctly for every AYN employee. New AYN employees may be added by clicking “Add New” Button and existing AYN employees may be deleted by selecting the the employee then clicking the “Delete” button. The time of which an AYN employee worked may be input by selecting the start minute and end minute of the AYN employee on the timeline. Data will save on clicking “Back”. An example is given below.

\bibliographystyle{plain}
\bibliography{dissertation}
\end{document}
